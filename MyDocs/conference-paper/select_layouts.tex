%%%%%%%%%%%%%%%%%%%%%%%%%%%%%%%%%%%%%%%%%%%%%%%%%%%%%%%%%%%%%%%%%%%%%%%%%%%%%%%%%%%%%%%%%%%%%%%%%%%%%%%%%%%%%%%%%%%%%%
%
% Text Style:
% - One sentence per line, no line breaks within one sentence
% - Use \tabref and \figref for table and figure references
% - Do not ``beautify'' the tex code with your editor
% - Use american english
% - Use <value>\,<unit>, e.g.,  1\,Hz, 40\,Watt
%

% Structure:
% - Intro: Why HDEEM?
% – Related work: What else is happening in the world that is related?
% - Challenges in Measuring Power and Energy: What ae the general challenges, what hurts accuracy and grnaularity?
% – Current Measurements: What did we achieve so far (HRSK2, phase 1)?
% – Integration in HPC env.: How did we integrate this into our tools?
% – Future Developments: What will the Newsca2 look like (specifications)?
%   What are the results of the current prototype?
%
%

% Commands:
%
% - Use your own command to define tasks for other people. E.g., \JS{RS, please write \ldots} should be used by Joseph Schuchart to assign work to Robert Schoene. If you do not have your own latex command, define it!
% - Use \tabref and \figref for table and figure references
% - Use \todo for general todos
% 

% Build:
% pdflatex hdeem.tex
% bibtex hdeem
% pdflatex hdeem.tex
% pdflatex hdeem.tex
% pdflatex hdeem.tex
%
%%%%%%%%%%%%%%%%%%%%%%%%%%%%%%%%%%%%%%%%%%%%%%%%%%%%%%%%%%%%%%%%%%%%%%%%%%%%%%%%%%%%%%%%%%%%%%%%%%%%%%%%%%%%%%%%%%%%%%

\documentclass[conference]{IEEEtran}

% correct bad hyphenation here

\usepackage{color}
\usepackage{listings}
\usepackage{pgfplots}
\usepackage[font=footnotesize]{subfig}
%\usepackage{graphicx,dblfloatfix}
\usepackage{stfloats}
\usepackage{url,hyperref}

% HKS 44_K 100%
% HKS 57_K 100%
% HKS 36_K 100%

\begin{document}
%
% paper title
% can use linebreaks \\ within to get better formatting as desired
\title{Multiparameter resources selection for next generation HPC platforms}


% author names and affiliations
% use a multiple column layout for up to three different
% affiliations
\author{\IEEEauthorblockN{Yiannis Georgiou, David Glesser, \\Dineshkumar Rajagopal}
\IEEEauthorblockA{BULL S.A.S\\
%01062, Dresden, Germany\\
Email: \{yiannis.georgiou, david.glesser, \\dineshkumar.rajagopal\}@bull.net}\\
%weird hack
~~~~~~~~~~~~~~~~~~~~~~~~~~~~~~~~~~~~~~~~~
\and
\IEEEauthorblockN{Matthieu Hautreux}
%TODO Bull please fill in here
\IEEEauthorblockA{CEA DAM
\\
Email: \{matthieu.hautreux\}@cea.fr}\\
%weird hack
~~~~~~~~~~~~~~~~~~~~~~~~~~~~~~~~~~~~~~~~~~~~

}
% make the title area
\maketitle


\newcommand{\JS}[1]{\textcolor{blue}{[JS: #1]}}
\newcommand{\TI}[1]{\textcolor{red}{[TI: #1]}}
\newcommand{\RS}[1]{\textcolor{cyan}{[RS: #1]}}
\newcommand{\D}[1]{\textcolor{violet}{[DH: #1]}}
\newcommand{\RED}[1]{\textcolor{red}{[ #1]}}
\newcommand{\YG}[1]{\textcolor{orange}{[ #1]}}

\newcommand{\figref}[1]{Figure~\ref{#1}}
%\newcommand{\figref}[1]{\underline{Figure~\ref{#1}}}

\newcommand{\tabref}[1]{Table~\ref{#1}}

\newcommand{\secref}[1]{Section~\ref{#1}}

\newcommand{\todo}[1]{\textbf{\textit{TODO: {#1}}}}

\renewcommand{\bottomfraction}{.5}

\setlength{\itemsep}{-2pt}

\begin{abstract}
SLURM(Simple Linux Utility for Resource Management) is a Resource and Job Management System(RJMS) in the High Performance Computing(HPC) System software stack. It is a middleware system software between User applications and Operating system  to distribute the HPC Computing resources(Nodes) to the users Job requirements(Constraints) effectively. Computing nodes internal architecture is evolving and having many-socket,many-core and multi-threaded features to increase the HPC and user applications performance and throughput. SLURM select/cons\_res resource selection plugin consuming nodes internal resources of cores,sockets,memory,GPUs to satisfy the users requirement. This will increase the performance and throughput of HPC environment, But resource selection is slow and compute nodes internal architecture information from minimum data. If we represent the internal nodes architecture informations and relationships between sub-resources, than the resource selection will not be manageable for the future architecture evolutions and introduction of new types of resources. To Resolve that by using general resource management framework called LAYOUTS in slurm to represent informations,relations to manipulate physical(e.x Node) and virtual(e.x Rack, Room, etc.) resources in HPC. 

%\hspace{0.5cm}
SLURM(Simple Linux Utility for Resource Management) is a Resource and Job Management System(RJMS) in the High Performance Computing(HPC) System software stack. It is a middleware system software between User applications and Operating systems  to distribute the HPC Computing resources(Nodes) to the user's Job requirements(Constraints) effectively. Computing nodes internal architecture is evolving and having many-socket,many-core and multi-threaded features to increase the HPC and user applications performance and throughput. SLURM select/cons\_res resource selection plugin consuming nodes internal resources of cores,sockets,memory,GPUs to satisfy the users requirement. This will increase the performance and throughput of HPC environment. Current resource selection(cons\_res) plugin in the SLURM is slow and not easy to adapt for the new HPC and Compute nodes architecture, Because of using linear data structure(Bitstring) and having limited data and relationships between HPC resources and sub-resources. If we represent the internal nodes architecture informations and relationships between sub-resources for current Intel based cluster, than the resource selection will not be maintainable for the future architectural evolution and new type of resources(GPUs,MICs,Intel Xeon Phi,Quantum Computers). To Resolve that by using general purpose resource management framework called LAYOUTS in the slurm to represent complete informations and relationships between resources to manipulate physical(e.x Node, Switch, etc.) and virtual(Simply anything e.x Rack, Room, etc.) resources in the HPC for different architectures easily. LAYOUTS framework used tree data structure to represent relationships between resources and sub-resources, So it improved the performance of basic operations to sub-linear and improved the performance of cons\_res plugin performance drastically/considerably.
\end{abstract}

% For peer review papers, you can put extra information on the cover
% page as needed:
% \ifCLASSOPTIONpeerreview
% \begin{center} \bfseries EDICS Category: 3-BBND \end{center}
% \fi
%
% For peerreview papers, this IEEEtran command inserts a page break and
% creates the second title. It will be ignored for other modes.
\IEEEpeerreviewmaketitle

\section{Introduction}
RJMS is complicated and changed by the changes in the HPC hardware and application requirements. RJMS has to support dynamical changes of the resources and easy to maintain. Current resource management does not support that flexibility and resource selection is not getting complicated for the future HPC.  Resource selection plugin managing HPC resources and allocating best resources to the user's job based on the different criteria(number-of-tasks, core, memory, disk-space, GPU, power and temperature). Parameter(criteria) for resource selection is changing over the decades. For an example power and temperature is not a criteria in the Peta-scale HPC system, But it is an important factor in the future Exa-scale systems. So our resource manager should support the changes. Consumable resource(cons\_res) selection plugin is developed for Peta-Scale systems to consume resources within the single node to improve the performance and throughput in the HPC system, But cons\_res plugin does not manage resources information best manner. Lack of resources information management complicated resource selection, So it is not easy to add a new criteria to the resource selection with the current plugin only flexibility mechanism. Previous cons\_res plugin limitations will be resolved By using LAYOUTS framework. Current LAYOUTS framework implemented in the SLURM to support generic resources(physical,virtual,simply anything) and relations to support future HPC resources and architectural changes. Selection plugin left the resources information to the LAYOUTS framework and having only resource selection algorithm to select the best resources based on the different parameters(criteria). LAYOUTS managing resources informations and relationships between the resources in the central place to avoid resources information in the different plugins locally. Currently it supports tree relationship between resources,But multi-tree and graph relationships to make the LAYOUTS framework to represent relationship between resources without  any restrictions. LAYOUTS framework internal details will be explained in more details in \textbf{Section III}.


%\vspace{-.5mm}
\section{Related Work}
\label{sec:related_work}


\section{Layouts Framework}


\section{Resources selection improvement}

\section{Experimentation and Performance Evaluation}

\label{sec:future}



\section{Conclusion and Future Work}





% The "triggered" command can be changed if desired:
%\IEEEtriggercmd{\enlargethispage{-5in}}

% references section

\bibliographystyle{IEEEtran}
\bibliography{references_all}





% that's all folks
\end{document}


