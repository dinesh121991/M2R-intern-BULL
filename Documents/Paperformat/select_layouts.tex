%%%%%%%%%%%%%%%%%%%%%%%%%%%%%%%%%%%%%%%%%%%%%%%%%%%%%%%%%%%%%%%%%%%%%%%%%%%%%%%%%%%%%%%%%%%%%%%%%%%%%%%%%%%%%%%%%%%%%%
%
% Text Style:
% - One sentence per line, no line breaks within one sentence
% - Use \tabref and \figref for table and figure references
% - Do not ``beautify'' the tex code with your editor
% - Use american english
% - Use <value>\,<unit>, e.g.,  1\,Hz, 40\,Watt
%

% Structure:
% - Intro: Why HDEEM?
% – Related work: What else is happening in the world that is related?
% - Challenges in Measuring Power and Energy: What ae the general challenges, what hurts accuracy and grnaularity?
% – Current Measurements: What did we achieve so far (HRSK2, phase 1)?
% – Integration in HPC env.: How did we integrate this into our tools?
% – Future Developments: What will the Newsca2 look like (specifications)?
%   What are the results of the current prototype?
%
%

% Commands:
%
% - Use your own command to define tasks for other people. E.g., \JS{RS, please write \ldots} should be used by Joseph Schuchart to assign work to Robert Schoene. If you do not have your own latex command, define it!
% - Use \tabref and \figref for table and figure references
% - Use \todo for general todos
% 

% Build:
% pdflatex hdeem.tex
% bibtex hdeem
% pdflatex hdeem.tex
% pdflatex hdeem.tex
% pdflatex hdeem.tex
%
%%%%%%%%%%%%%%%%%%%%%%%%%%%%%%%%%%%%%%%%%%%%%%%%%%%%%%%%%%%%%%%%%%%%%%%%%%%%%%%%%%%%%%%%%%%%%%%%%%%%%%%%%%%%%%%%%%%%%%

\documentclass[conference]{IEEEtran}

% correct bad hyphenation here

\usepackage{color}
\usepackage{listings}
\usepackage{pgfplots}
\usepackage[font=footnotesize]{subfig}
%\usepackage{graphicx,dblfloatfix}
\usepackage{stfloats}
\usepackage{url,hyperref}

% HKS 44_K 100%
% HKS 57_K 100%
% HKS 36_K 100%

\begin{document}
%
% paper title
% can use linebreaks \\ within to get better formatting as desired
\title{Multiparameter resources selection for next generation HPC platforms}


% author names and affiliations
% use a multiple column layout for up to three different
% affiliations
\author{\IEEEauthorblockN{Yiannis Georgiou, David Glesser, \\Dineshkumar Rajagopal}
\IEEEauthorblockA{BULL S.A.S\\
%01062, Dresden, Germany\\
Email: \{yiannis.georgiou, david.glesser, \\dineshkumar.rajagopal\}@bull.net}\\
%weird hack
~~~~~~~~~~~~~~~~~~~~~~~~~~~~~~~~~~~~~~~~~
\and
\IEEEauthorblockN{Matthieu Hautreux}
%TODO Bull please fill in here
\IEEEauthorblockA{CEA DAM
\\
Email: \{matthieu.hautreux\}@cea.fr}\\
%weird hack
~~~~~~~~~~~~~~~~~~~~~~~~~~~~~~~~~~~~~~~~~~~~

}
% make the title area
\maketitle


\newcommand{\JS}[1]{\textcolor{blue}{[JS: #1]}}
\newcommand{\TI}[1]{\textcolor{red}{[TI: #1]}}
\newcommand{\RS}[1]{\textcolor{cyan}{[RS: #1]}}
\newcommand{\D}[1]{\textcolor{violet}{[DH: #1]}}
\newcommand{\RED}[1]{\textcolor{red}{[ #1]}}
\newcommand{\YG}[1]{\textcolor{orange}{[ #1]}}

\newcommand{\figref}[1]{Figure~\ref{#1}}
%\newcommand{\figref}[1]{\underline{Figure~\ref{#1}}}

\newcommand{\tabref}[1]{Table~\ref{#1}}

\newcommand{\secref}[1]{Section~\ref{#1}}

\newcommand{\todo}[1]{\textbf{\textit{TODO: {#1}}}}

\renewcommand{\bottomfraction}{.5}

\setlength{\itemsep}{-2pt}

\begin{abstract}



\end{abstract}

% For peer review papers, you can put extra information on the cover
% page as needed:
% \ifCLASSOPTIONpeerreview
% \begin{center} \bfseries EDICS Category: 3-BBND \end{center}
% \fi
%
% For peerreview papers, this IEEEtran command inserts a page break and
% creates the second title. It will be ignored for other modes.
\IEEEpeerreviewmaketitle



\section{Introduction}



%\vspace{-.5mm}
\section{Related Work}
\label{sec:related_work}


\section{Layouts Framework}


\section{Resources selection improvement}

\section{Experimentation and Performance Evaluation}

\label{sec:future}



\section{Conclusion and Future Work}





% The "triggered" command can be changed if desired:
%\IEEEtriggercmd{\enlargethispage{-5in}}

% references section

\bibliographystyle{IEEEtran}
\bibliography{references_all}





% that's all folks
\end{document}


